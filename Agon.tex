\documentclass{article}

\usepackage{alltt}

\newcommand\void{\textsc{Void}}

\newenvironment{stanza}{\begin{alltt}\normalfont}{\end{alltt}}

\title{Agon}

\begin{document}

\maketitle

\clearpage

%Umberto Eco: The Name of the Rose

\begin{stanza}
I see now through a glass darkly
The truth in scattered fragments
In the errors of this, our world,
We must spell out its signals
Even when they seem obscure,
With a will bent on suffering.

The monk must rise in darkness
And pray at length in darkness,
Waiting for the light of day,
Illuminating the shadows
With an insane devotion,
With the zeal of a martyr,
Or the arrogance of the damned,
With his face hidden in hands,
At distance from the carnal life.

Dressed in torn and dirty habits,
All bow toward the altar
In a moment of deep thought,
In these hours of mystic ardour,
Of intense inner peace and bliss
That no-one else can comprehend.

I am waiting to be lost
In the bottomless abyss
Of silence and divinity
Where there is no work, no image,
Where man is reduced to smoke,
Dispersed by the strong wind of faith.

I have been taught to recognise
The signs through which the cryptic earth
Speak to us like a great book.

Dictated to our prophets,
Custodians of the void,
Without changing a single word,
It was the order to oppose
Our futile race to the abyss.

Objects of wondrous beauty
Shine in the glow of the sun
As shapes of things and animals
Suddenly rise from the void.

The weak sun is beating straight down
And plunges my drugged mind
Into a violent vision
That my tongue can hardly describe.

My eyes are wide and glaring
Over a telluric mankind
That matured and finally reached
The end of its cruel story.

As if his shoulders and neck
Were twisted in a fierce impulse,
With flanks tensed and twitching muscles,
Limbs of a dying animal,
Serpent-like tail coiled and writhing,
Culminating at the top,
In pillars of clouds and fire.

We see the final results
Of the universal slaughter,
Stretched out to its full extent
The hide of a skinned animal,
Its sinews, the viscera,
And its petrified organs,
And even features of the face,
The pus dripping from the heel,
The thin threads of the lashes,
The runny substance of the eyes,
The tender flesh of the lips,
The delicate spine of the back,
The fragile structure of the bones,
The flesh lying flat like a robe,
The embroidery of the veins,
The scarlet pile of viscera,
The tongue as a blue pendant,
The carpet of the belly,
The intense ruby of the heart,
No longer pumping potent drugs,
Everything reduced to powder.

Their bodies are inhabited
In every part by the void,
Struck by potent revelation,
Faces painted with apathy,
Their eyes shining with fervour,
Pupils dilated by zeal,
Bony hands raised to the sky,
With fingers splayed like empty wings,
Howling their own damnation
Like the sound of many waters
From throats obscene and barren,
Roaring in adoration
Of what is to come upon us.

The son will slay his father,
With liquor gurgling in his throat,
Overturning his position
With respect to his former lord,
The servant feels he is master,
No-one will respect the old,
The young will demand to rule,
With dissolute behaviour,
Work will seem a useless chore,
Sins against nature will follow.

Supreme bliss in having lost all,
A champion of poverty,
Perfect despiser of the world,
This is what I wish to become.

The radiant sun above
Invites the spirit of man
To lose all memory in bliss,
That extends and enlarges man
That extends the gaping chasm
That he bears within himself
Is no longer easily sealed,
A wound cut by the living void.

I am weary and detached
From the wretched things of this world,
Formerly driven by hope,
Refusing to recognise
The world as a vale of tears
Where even smallest injustice
Is foreordained by providence
To maintain the subtle balance
Of a profoundly hostile scheme
Whose design we cannot fathom.

Bedecked with fragrant flowers,
Beauty is ever fleeting
And must be considered base,
At once mother, traitor and whore,
Screaming and railing in lust
Like a poisonous snake in heat.

As my body grows older
And as I abandon myself
To the will of the living void,
As life gradually narrows,
The sole element of reason
Is the will to endure, to wait,
Fully immersed in gloom and doom.

Without the sacred weapons
That can subdue and tame the mind
The soul will tumble and plunge
Through existential terrors
Into an immense abyss
For we are nothing without fear,
The most foresighted of our gifts
That restores the mind of man
To its original form,
For nothing exists that so fills
And binds the heart as fear does.

In the transparent foundation,
In the mute and distant desert
Where diversity is not known,
Beautiful and terrible
As an army arrayed for war,
This is where I shall be found.

The people have been transformed
Into an assembly belched forth
From the vast abysms of the earth,
Rewarded and liberated
From the seduction of passions,
Consigned to the eternal frost,
The triumph of corruption.

We can only name the void
Through the most distorted things,
On this side the choir of man,
On that the gaping emptiness.
\end{stanza}

%Charles Baudelaire - The Flowers of Evil

\begin{stanza}
The inward voice has vanished
With the sound of its savage moan,
With its drone that used to wail
Amidst a haze of dirty scants
Against the varied display
Of a sick and worthless age
In which, vicious and minute,
Our human races fester.

We leave this, our fading world,
To defunct offspring, to the void
And step forth on the blood red soil.

Wandering in the gutter
With the dread of a dying sot,
Beneath an erratic sky,
Mumbling with a jaded voice,
Looking upon high, full of fear,
Heaven above, oppressive wall,
There lies but the joy that destroys,
Void the purse and void the palace,
The dream of an eremite.
\end{stanza}

\iffalse
\fi

\end{document}
