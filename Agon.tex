\RequirePackage{lineno}

\documentclass{article}

\newcommand\void{\textsc{Void}}

\newenvironment{stanza}{\begin{minipage}{10cm}\begin{internallinenumbers}\obeylines}{\end{internallinenumbers}\end{minipage}\vspace{\baselineskip}}

\title{Agon}

\begin{document}

\maketitle

\clearpage

%Umberto Eco: The Name of the Rose

\begin{stanza}
I see now through a glass darkly
The truth in scattered fragments
In the errors of this our world,
We must spell out its signals
Even when they seem obscure,
With a will bent on suffering.
\end{stanza}

\begin{stanza}
The monk must rise in darkness
And pray at length in darkness,
Waiting for the light of day,
Illuminating the shadows
With an insane devotion,
With the zeal of a martyr,
Or the arrogance of the damned,
With his face hidden in his hands,
At distance from the carnal life.
\end{stanza}

\begin{stanza}
Dressed in torn and dirty habits,
All bow toward the altar
In a moment of deep thought,
In these hours of mystic ardour,
Of intense inner peace and bliss
That no-one else can comprehend.
\end{stanza}

\begin{stanza}
I am waiting to be lost
In the bottomless abyss
Of silence and divinity
Where there is no work, no image,
Where man is reduced to smoke,
Dispersed by the strong wind of faith.
\end{stanza}

\begin{stanza}
I have been taught to recognise
The signs through which the cryptic earth
Speak to us like a great book.
\end{stanza}

\begin{stanza}
Dictated to our prophets,
Custodians of the void,
Without changing a single word,
It was the order to oppose
Our futile race to the abyss.
\end{stanza}

\begin{stanza}
Objects of wondrous beauty
Shine in the glow of the sun
As shapes of things and animals
Suddenly rise from the void.
\end{stanza}

\begin{stanza}
The weak sun is beating straight down
And plunges my drugged mind
Into a violent vision
That my tongue can hardly describe.
\end{stanza}

\begin{stanza}
My eyes are wide and glaring
Over a telluric mankind
That matured and finally reached
The end of its cruel story.
\end{stanza}

\begin{stanza}
As if his shoulders and neck
Were twisted in a fierce impulse,
With flanks tensed and twitching muscles,
Limbs of a dying animal,
Serpent-like tail coiled and writhing,
Culminating at the top,
In pillars of clouds and fire.
\end{stanza}

\begin{stanza}
We see the final results
Of the universal slaughter,
Stretched out to its full extent,
The hide of a skinned animal,
Its sinews, the viscera,
And its petrified organs,
And even features of the face,
The pus dripping from the heel,
The thin threads of the lashes,
The runny substance of the eyes,
The tender flesh of the lips,
The delicate spine of the back,
The fragile structure of the bones,
The flesh lying flat like a robe,
The embroidery of the veins,
The scarlet pile of viscera,
The tongue as a blue pendant,
The carpet of the belly,
The intense ruby of the heart,
No longer pumping potent drugs,
Everything reduced to powder.
\end{stanza}

\begin{stanza}
Their bodies are inhabited
In every part by the void,
Struck by potent revelation,
Faces painted with apathy,
Their eyes shining with fervour,
Pupils dilated by zeal,
Bony hands raised to the sky,
With fingers splayed like empty wings,
Howling their own damnation
Like the sound of many waters
From throats obscene and barren,
Roaring in adoration
Of what is to come upon us.
\end{stanza}

\begin{stanza}
The son will slay his father,
With liquor gurgling in his throat,
Overturning his position
With respect to his former lord,
The servant feels he is master,
No-one will respect the old,
The young will demand to rule,
With dissolute behaviour,
Work will seem a useless chore,
Sins against nature will follow.
\end{stanza}

\begin{stanza}
Supreme bliss in having lost all,
A champion of poverty,
Perfect despiser of the world,
This is what I wish to become.
\end{stanza}

\begin{stanza}
The radiant sun above
Invites the spirit of man
To lose all memory in bliss,
That extends and enlarges man
That extends the gaping chasm
That he bears within himself
Is no longer easily sealed,
A wound cut by the living void.
\end{stanza}

\begin{stanza}
I am weary and detached
From the wretched things of this world,
Formerly driven by hope,
Refusing to recognise
The world as a vale of tears
Where even smallest injustice
Is foreordained by providence
To maintain the subtle balance
Of a profoundly hostile scheme
Whose design we cannot fathom.
\end{stanza}

\begin{stanza}
Bedecked with fragrant flowers,
Beauty is ever fleeting
And must be considered base,
At once mother, traitor and whore,
Screaming and railing in lust
Like a poisonous snake in heat.
\end{stanza}

\begin{stanza}
As my body grows older
And as I abandon myself
To the will of the living void,
As life gradually narrows,
The sole element of reason
Is the will to endure, to wait,
Fully immersed in gloom and doom.
\end{stanza}

\begin{stanza}
Without the sacred weapons
That can subdue and tame the mind
The soul will tumble and plunge
Through existential terrors
Into an immense abyss
For we are nothing without fear,
The most foresighted of our gifts
That restores the mind of man
To its original form,
For nothing exists that so fills
And binds the heart as fear does.
\end{stanza}

\begin{stanza}
In the transparent foundation,
In the mute and distant desert
Where diversity is not known,
Beautiful and terrible
As an army arrayed for war,
This is where I shall be found.
\end{stanza}

\begin{stanza}
The people have been transformed
Into an assembly belched forth
From the vast abysms of the earth,
Rewarded and liberated
From the seduction of passions,
Consigned to the eternal frost,
The triumph of corruption.
\end{stanza}

\begin{stanza}
We can only name the void
Through the most distorted things,
On this side the choir of man,
On that the gaping emptiness.
\end{stanza}

%Charles Baudelaire - The Flowers of Evil

\begin{stanza}
The inward voice has vanished
With the sound of its savage moan,
With its drone that used to wail
Amidst a haze of dirty scants
Against the varied display
Of a sick and worthless age
In which, vicious and minute,
Our human races fester.
\end{stanza}

\begin{stanza}
Wandering in the gutter
With the dread of a dying sot,
Beneath an erratic sky,
Mumbling with a jaded voice,
Looking upon high, full of fear,
Heaven above, oppressive wall,
There lies but the joy that destroys;
Void the purse and void the palace,
The dream of an eremite.
\end{stanza}

\begin{stanza}
We leave this our fading world,
To defunct offspring, to the void
And step forth on the blood red soil.
\end{stanza}

%William S. Burroughs: Junkie, Queer, Cities of the Red Night, Exterminator!, The Yage Letters

\begin{stanza}
The ground shifting underfoot,
I am at the epicentre
Of the vast web that is the earth,
Vibrating with life like a sun;
Suddenly, I know its purpose,
The reason for suffering,
For fear, sex, violence and death;
All of it is intended
To keep us, the human victims,
Caged in our physical bodies,
Unable to attain peace,
Through revelation of the void.
\end{stanza}

\begin{stanza}
Every night, in every city,
The people will be uglier,
On a nauseating speed-up
Into a solar vortex,
Of bodily dissolution
And incessant, meaningless change.
\end{stanza}

\begin{stanza}
Great waves of hostility
And distrust flow out from my eyes;
My head and face bear the marks
Of a constant losing fight;
Viscera and cells galvanised
Into an insect-like being,
On the point of breaking through
The remains of the human shell,
Drifting in islands of rubbish,
Self-image disintegrating,
Indifferent to the world,
Completely unsure of myself
And my purpose here on earth;
I sit down shattered, face in hands.
\end{stanza}

\begin{stanza}
My facial traits are distorted,
Vague, blurred, even tumescent;
Fear and suspicion are written
Across a face that most people
Instantly dislike or distrust
And my silence seems to hold
A vicious threat or criticism
Against all of life, all of death.
\end{stanza}

\begin{stanza}
Where life withers and withdraws
Death moves in and takes its place
In a cryptic universe
That no-one in his senses
Would ever trust or affirm.
\end{stanza}

\begin{stanza}
Across all skies of this our world,
Behold the silent writing,
Hear my calls looking for death;
All words cancelled forever.
\end{stanza}

%Hermann Hesse: Siddhartha

\begin{stanza}
I see mankind going through life
In a simple, childlike manner,
In an animal-like manner,
Which I adore and despise,
Love and hate, at the same time.
\end{stanza}

\begin{stanza}
You complain about minor pains
At which he would only smile,
You suffer deprivations
Which he would not even feel.
\end{stanza}

%Emil Cioran: On the Heights of Despair, Tears and Saints, A Short History of Decay, The Temptation to Exist, The Trouble with Being Born, All Gall Is Divided, The New Gods, Drawn and Quartered, Anathemas and Admirations, The Fall into Time, History and Utopia

\begin{stanza}
To see death spread over this world,
How it kills a tree or a bird,
How it penetrates our dreams,
How it withers a flower
Or a civilisation,
How it gnaws on a person
Like a destructive cancer;
It means to be beyond tears
Beyond regrets, system and form,
With a drive, an affinity
For suffering and silence,
Ashamed of being human,
Hanging over the abyss
Of existential nausea
Where ideals are declared void.
\end{stanza}

\begin{stanza}
We are slaves to senseless work,
Toiling for our defunct offspring,
For those generations to come,
Under the dire delusion
That we created a future
For the good of humanity;
Suddenly, we avenge ourselves
On the mediocrity
Of a sterile existence,
Of an insignificant life,
On the immense waste of being
That had never permitted
Creative transfiguration.
\end{stanza}

\begin{stanza}
I hold that one's life must beat
To an essential rhythm,
An experience so intense
That it synthesises you
And your personality.
\end{stanza}

\begin{stanza}
When one's glands and one's instincts 
Begin to serve another world,
Rising like a surge of blood,
Like the choking grasp of a snake
Which provokes hallucinations,
One renounces humanity,
Even if it means solitude,
Living in flames and constant pain.
\end{stanza}

\begin{stanza}
He who never errs out of pain
And only knows love and laughter,
Is so colourless and sterile
That not even the sun above
Can brighten his dismal gestalt.
\end{stanza}

\begin{stanza}
The wise man's life is empty,
For it is free from despair,
It is free from contradictions;
So much richer and creative,
So much more complex is the man
Who is constantly suffering
From limitless anxiety,
A mind devoid of earthly love,
With thoughts fed on suffering.
\end{stanza}

\begin{stanza}
Is it not tragic to be man,
That dissatisfied animal
Suspended between life and death,
That is suffering aimlessly,
Lost in the uselessness of pain,
Mirroring itself in its blood,
Masking its inner torments,
Wallowing in inner deserts.
\end{stanza}

\begin{stanza}
The rocky road to ecstasy,
The experience of the void,
The strife for mental blankness,
The strife for total rejection
Of the images of the world.
\end{stanza}

\begin{stanza}
Heroism means transcending life,
A fatal leap into the void
That is being born around you;
It occupies territory
Deserted by space and time,
Emptied of vitality
Where nothing but thorns can bloom.
\end{stanza}

\begin{stanza}
Whoever has overcome
His fear of suffering and death
Has also triumphed over life,
Which is just another word
For that same fear, an expression
Of non-participation,
An ecstasy that destroys
The wretched fruit of this our world.
\end{stanza}

\begin{stanza}
I try to keep my eyes half-closed,
So that the lids can protect
The sacred inner mystery
From the profane indiscretion
Of the light of the dying sun
That has drowned my distant dreams
Of inner peace and silence.
\end{stanza}

\begin{stanza}
You prove the health of your senses
To the extent that you resist
The bliss of asceticism,
The negation of life itself
Through permanent lucidity,
A theme for meditation.
\end{stanza}

\begin{stanza}
Ill-equipped for happiness,
A face devoid of emotions
Matching the ambiguity
Of an utterly insane stare,
The boredom preceding the joy,
The sadness succeeding the joy
That never was, never will be.
\end{stanza}

\begin{stanza}
Despair, more than any feeling,
Establishes, between being
And our abhorred environment,
Through tears, sighs and detachment
An intimate correspondence.
\end{stanza}

\begin{stanza}
Now, I see the futility
In loving transitory things,
The painful fall from ecstasy,
Rediscovering the body,
Conductor of disillusions.
\end{stanza}

\begin{stanza}
Faith is a cherished mechanism
Of the weak and desperate mind
To avoid the inner torments
Of solitary monologue;
It must trample under foot
All intellect, reason and doubt.
\end{stanza}

\begin{stanza}
The practical achievements
Of our affection and our love
Are absolutely nothing
But an immense illusion
Perpetuated by cultures
Since the very dawn of our race.
\end{stanza}

\begin{stanza}
The greatest changes in thought
Are born from the revelation
Of the meaninglessness of life,
The profound realisation
That the bliss of the absolute
Lies in renunciation.
\end{stanza}

\begin{stanza}
It is not a will to power,
It is the will to nothingness,
An agonistic passion
Whose devious poison touch
Has spread over centuries.
\end{stanza}

\begin{stanza}
This is the stony despair
Of an ascetic who has failed,
Who discovers too late in life
The pointlessness of renouncing,
Who tried to conquer and defeat
The temptations of the world
Through hardship and crippling fear.
\end{stanza}

\begin{stanza}
The world is mediocre,
Perhaps even superfluous
And as my time in it runs out,
I am removed and thrown
Into a vibrant nothingness
Full of noise and vague promises,
The home of the void we seek,
The only art capable
Of muting the intellect,
Of bringing comfort and sleep.
\end{stanza}

\begin{stanza}
My weariness of life fares well
In shadows of whores and hermits
Yet unable to receive
The revelations of the void,
For too great is my love of life,
Seething with cosmic hatred
Against all agents of cultures
That are not of emptiness,
That are foreign, that are not mine.
\end{stanza}

\begin{stanza}
The more you advance in life,
The more you acknowledge defeat,
You see that you learned nothing,
Yet still, you are not wise enough
Not to love your existence
With infinite agony.
\end{stanza}

\begin{stanza}
At the end of a regression
Inside our emetic selves,
When finally confronted
With the face of the universe,
The irrelevance of being
Would crush our feeble spirits,
It would maim all our earthly joys.
\end{stanza}

\begin{stanza}
My sexual orgasms pale
Beside the ecstatic trance
Of my drugged insane ideals,
A convenient refuge from thoughts
Of individuality.
\end{stanza}

\begin{stanza}
Free time is a bearable curse
For a solitary thinker,
But for others it is torture;
The starting point is boredom,
The end is self-destruction;
Any contact with this earth
Adds to the sum of my despair,
It adds to the tiring sickness
That brings grace and nourishes
My otherworldly passions.
\end{stanza}

\begin{stanza}
To satisfy one's desire
To be distant, to be alone
In noise, in a society
That rebels against its pain,
Despite having realised
One cannot suffer for others.
\end{stanza}

\begin{stanza}
The more time has died and vanished
From the bowels of memory,
The closer one is to the void,
To the disfigured landscapes,
To the eternal struggle
Between form and emptiness.
\end{stanza}

\begin{stanza}
After realising in man
All the possibilities
For corruption, you begin
By ignoring the object
And end by ignoring the world;
Who will carry the burden
Of all my deficiencies?
\end{stanza}

\begin{stanza}
We are degenerate dreamers
Who happen to be entertained
By massacres and genocides
And yet we refuse to admit
The interchangeable nature
Of mind, blood, form and emptiness.
\end{stanza}

\begin{stanza}
Surrender your humanity
To the void raging within,
To a mind suffocating
Within the narrow confines
Of the wretched world of man,
That leaves you to your apathy,
To your despair and uselessness.
\end{stanza}

\begin{stanza}
Youthful desires are burning
In a sickly conflagration
In which women are reduced
To the role of motherhood
Or equally futile whoredom.
\end{stanza}

\begin{stanza}
When matter defies man's will
His eyes look up and see nothing
But a system of mistakes,
It is the universal yawn.
\end{stanza}

\begin{stanza}
I sleep with eyes open towards
The incomprehensible,
With irreconcilable ways
Of looking at our wretched world
And the emetic race of man.
\end{stanza}

\begin{stanza}
My body and mind are plunged
Into an impenetrable
And unsustainable world,
In a mass of desires
And convictions superimposed
On our fading reality
Like a morbid structure.
\end{stanza}

\begin{stanza}
There is no-one more dangerous
Than he who has suffered greatly
For a belief, a conviction
So strong that it is capable
Of replacing reality.
\end{stanza}

\begin{stanza}
Great sufferers are never bored,
Potent drugs and disease fill them,
In the same way that remorse
Feeds and binds the great criminals.
\end{stanza}

\begin{stanza}
The frequent attempts of mankind
To compromise and diminish
The prestige of death and decay
Shows the insanity of life
For only impurity
Is a sign of reality.
\end{stanza}

\begin{stanza}
The misery of the glands,
Disease, is an activity,
The most intense, the most extreme
A person can indulge in.
\end{stanza}

\begin{stanza}
Reality is created
From mankind's disproportions
From the worst of our excesses,
From our emetic spectacle,
From our mental derangements,
From our struggle with disease;
All our convulsions propose
A terrible reality
To the human consciousness,
Which nobody can elude.
\end{stanza}

\begin{stanza}
I developed the compulsion
To preach formulas for being,
The revelation of the void
To all of humanity;
Every faith utilises
Some form of coercion, terror.
\end{stanza}

\begin{stanza}
To be fooled, to live and die duped,
That is certainly what men do
In this our atrocious world
That can take everything from us
Where man is like a crawling worm
Upon the cosmic carrion.
\end{stanza}

\begin{stanza}
The world seems to have infested
My health, my sacred solitude,
In which I was daily engulfed
In potent intoxications,
I which I constantly profaned
The things that are born and die
Under the moon and the sun.
\end{stanza}

\begin{stanza}
I see that life is everything,
A state of non-suicide
That modulates and refines
The monotony of being
In a world where we exist 
Only as we breathe and suffer.
\end{stanza}

\begin{stanza}
Things and words lie before us,
Unclean, like verbal carrion
And all of them carry within
The germs of disappointment,
Glowing and shining in the light
Of their unreality.
\end{stanza}

\begin{stanza}
To survive the comprehension
Of our universal grief,
As suffering emanates
From our viscera and glands
And joins the void triumphant.
\end{stanza}

\begin{stanza}
Each being feeds on the torment,
The agony of some other;
Life's brutal ambiguity
Requires impulse over wit,
Submission over resistance.
\end{stanza}

\begin{stanza}
To be liberated from sweat,
To be freed from toiling in vain,
To have the authority
To question the conditions
Of one's very existence,
To admit that every pain
Is capable of replacing
One's sense of reality.
\end{stanza}

\begin{stanza}
Under the goad of suffering,
The flesh rises and awakens,
It groans in lucid substance
And sings its its dissolution.
\end{stanza}

\begin{stanza}
We were taught to laud and respect
The falsehoods of some culture,
The context of our ordeals,
Forged by ages of decadence.
\end{stanza}

\begin{stanza}
I realised the non-meaning
Of all action and effort
But who is ever bold enough
To be idle, to do nothing
Because action is senseless
When facing infinity.
\end{stanza}

\begin{stanza}
Alas, I fear it is too late
To release humanity
From the illusion of action,
It is too late for man
To be raised to the sanctity
Of idleness and lethargy.
\end{stanza}

\begin{stanza}
A mere proliferation
Of empty words and phrases,
Subtle displacements of meanings,
Yet nature, red in tooth and claw,
Rejects our verbal embrace.
\end{stanza}

\begin{stanza}
I have a lucid scorn for things,
Repulsion for the seeds of life,
For all that grows in the warmth
Of cultural illusion.
\end{stanza}

\begin{stanza}
To struggle with the unknown,
To be a slave to one's thoughts,
To promise the entirety
Of the things under the sun,
And to accomplish nothing.
\end{stanza}

\begin{stanza}
Can the fever thrive and triumph
Over our existential fears,
Over the disgusting odours
Emitted by a universe
Unworthy of the mind's perfume.
\end{stanza}

\begin{stanza}
The earth is a great slaughterhouse,
It is a fictive paradise
Where birth and chain are synonyms,
Where man is shackled and bound.
\end{stanza}

\iffalse
Our mysteries and our griefs have leaked away into our dreams
The approach of disgust, the sensation that physiologically separates us from the world, shows how destructible is the solidity of our instincts or the consistency of our attachments
Is anything viler than to say yes to the world
The mind discovers sloth
This is life without objective, without fears, without laughter
Woe to the conqueror without a word to say
Recapitulating the general inanity of our history, one finds no other dignity than that of a literature of failure and an aesthete of bloodshed
This obsession with the last frontier, this progress in the void, involves the most dangerous form of sterility, it is panic before an object which is no longer an object, for its limits have been transcended
Man is an animal with retarded desires
To survey all objects and possess none
There is no progression in the notion of universal vanity, nor a conclusion
Humanity will blush to beget when it sees things as they are
The mind's erosion of existence
Access to the highest privilege, that of destroying himself
Flashes of incurable fear and doubts intersected by sighs
It is from all the moments fed on the inaccessible that his power comes to him
The science of tears, the scourges of the heart, the formal orgies, versed in all idioms
It takes a thousand ordeals to achieve detachment
The ideally unhappy being, liberated from all the principles of custom
To denounce material prosperity
Biologically obliged to the false
In a selfish dream state, no longer an object outside oneself, no reason for hate or love, a fall into the languid mud, circling the damned without a hell, reiterations of a zeal to perish
You deploy your talent as a tyrant, glorious or absurd, so the victim may one day become the executioner
Try to be free and you will die of hunger
In search of rare sensations, this is the face of decadence
Instincts eroded by conversation
Unlearn to look at things
Weary of words, we have the disastrous consummation of the early venture
Racing towards the final model of man, mute and naked
To feel the weight of the race, to assume all its solitude
Disdain for the routine of being, rushing towards the abyss that threatens it
The species appals him, he has nothing more to transmit
Having reached the gilded peaks of his disgusts, at the antipodes of creation, he has made his nothingness into a halo
To exploit one's disequilibrium with violence and skill
In vain you search for your model among human beings, from those who might have gone farther than you
Bitterness in understanding, in your mode of action, is the one fixed point in your oscillation between disgust and self-pity
I have no word on hand to designate my abyss
I rest in the shadow of your nullity
To detest action, the mother of all vices
The depth of sighs conceals a grimace
A thinker requires to dissociate himself from the world
To enthrone doubt, where the mind never penetrated, to shake the calm of stones
I dissolve by day and solidify at night
Attachments dissolve under the searching eye
Strangers to yourselves, in an anonymous hell, where we dance and jeer as we destroy ourselves
Diminished in the succession of days, in the inertia of misery
Cruelty and intolerance, these are the signs of life
With awe and jealousy my thoughts are turned, towards the desert fathers
Impatient to degrade myself, to identify with the gutter
The merest desire conceals a source of insanity
Like a frenzied sage, I am dead to to the world, and frantic against it, I invalidate my illusions
It has, like life, the excuse of proving nothing
For how many people was he the cause of disaster
By accumulating non-mysteries, by monopolising non-meanings, life inspires more dread than death
In the end you realise that your life is still lived out in the elements out of which the world is constituted
It is in order to beget new unrealities, to escape a universe too demonstrably the same
Having lived out and verified all the arguments against life, I have experienced its nakedness
You feel the importance of your progress toward negation, you feel you are less
You no longer have the right to expect on earth a fate modified by hope
You have left the order of the world, you look at it without recognition
The world's futile disasters act as a confirmation of an innate disenchantment
Malfunctions of our organs determine the fruitfulness of our minds
Each generation raises monuments to the executioners of the one which preceded it
It seems that humanity offers itself up to its conquerors, it seeks to be trampled under foot
The man who scorns everything bows to conventions only to repudiate them on the sly
Life is tolerable only by the degree of mystification we endow it with
The evolution of a civilisation, from agriculture to paradox
The essence of the world is the pleasure of being disappointed
The beneficiaries of your worship, the profiteers of your abandons
In the distractions of your delirium, you no longer meet your afflictions
The curse attached to acts
Every work ultimately turns against its author
Released from humanity to lodge yourself in being
Claim your mode of collapse
To broaden the sphere of one's diseases
Breathing too fast to grasp things in themselves or to expose their fragility
Distorted and disfigured by our panting
We may scorn it but it is the symbol of our hysteria, here on earth
The neutral dream of an existence without qualities
I am the lover of carrion
He who belongs to a civilisation cannot identify with the nature of the disease that undermines it
The tension of a will inexhaustibly diseased
Blazing a trail into uncertainty, stumbling through truths
May I be blinded by the chain of illusions
I must fling myself, with my eyes closed, into any shallow belief that will grant me the protection and the peace of the yoke
A routed iconoclast in search of routine and impersonality, half prostrated, in his singularity, he has become the last idol to smash
A fanatic without convictions
To think methodically, a man must, with the help of his deficiencies, forget himself, no longer be an integral part of his ideas
Alternately exalted and downcast, he turns his morose and dazzled eyes upon the void
To be aggressive in one's disappointments
The more one is dispossessed, the more intense our appetites and our illusions become
You embody the extremes to which we aspire without achieving them
Impatient to get drunk on your tears, to wallow in your humiliation, to feast on your griefs
Acosmic
No creature less anonymous
To fight, without a flag, against the whole world
My promotion to solitude
He knows whom to vilify or implore, whom to attack or to pray to
You address me in the speech of lassitude, offering annihilation
There is no peace for the man who cannot cultivate abandon
With what complacency do they display their suffering and open their wounds
Whatever share of being was dispensed to me is being eroded by ennui
The void in action, it ransacks brains and reduces them to a heap of fractured concepts
Know that I destroy nothing, I record the imminent, the thirst of a world which is cancelling itself out, upon the wreck of its appearances, racing towards the unknown, the incommensurable
In fits of lucidity I become a servant of the irrational
An experience infinitely intense and without substance
Defined by the values you repudiate, without the pomp of negation
Our contradictions guide us, stimulate us and kill us
I have no use for the hero's life, I do not attend to it, I do not even believe in it
No point of reference
He will therefore abandon the facts inflicted on him by an erudite and barren age
A self without a future, ecstasy on the frontier, litany and soliloquy of the void
What matter our sensations, we fling ourselves into a madness that is not sacred
You were taught to endure your misfortunes with dignity, to contemplate your nullity in silence
To capture the faculty of suffering without a murmur
Overrun on every side with the elegance to die without a struggle
Eager to cover myself in ignominy, I envy all who expose themselves to the sarcasm and the spittle of the world, miss no occasion for solitude
The solitary artist writes for himself or for a faceless public in an equally faceless epoch
Digging ever deeper in a single direction, despising the new
Mobilising all one's defects to produce a work that conceals oneself
The intimate perception of our nothingness is only thinly veiled by this miserable prison world, so rich in adversity
He enjoys the consolations of an existence without a horizon
He ceases to communicate with his life which he turns into an object
No one knows what he still reacts to
Fear is indispensable  to action and thought, transformed into a harmful principle
The beings that live before our eyes
In abstinence the individual is set above and below the human race
We praise his ferocity , his contempt for men, his vision of a world condemned
The void we glimpse, at the bottom of the words, evokes the one we grasp in things
I envy the man who takes his ease among words, who lives there naively, by reflex, neither questioning nor identifying them with signs, as if they corresponded to reality itself
I cannot envy the man who sees through them, discerns their depths, their nothingness
In futile search of the non-man, but he cannot exist, absolute lucidity is incompatible with the physical reality of our organs
Stepping back from the line of other beings, leaving the beaten paths of salvation, he innovates in order to support his reputation as an interesting animal
If life has falsified matter, he has falsified life
We turn sour only in the vicinity of man
The mind cannot save this old flesh whose corruption prospers before our eyes
Comparing your sterility to that of the desert, the material image of your void
Some make their way from affirmation to affirmation, forever applauding reality, regardless of its qualities, they accept the universe and are not ashamed to say so
The contagion of nothingness, the comfort of the abyss
You look at no one, without anxiety in your eyes, without haste in your gestures, the outside world has ceased to exist, you submit to every solitude
To have committed every crime but that of a being a father
I watch the hours flow, independent of any reference, any action, any event, the disjunction of time from what is not itself, its empire, its tyranny
Confronting the empty universe
I need an alternate vision, with days of miraculous sterility and instead of celebrating them as a sign of my detachment and my maturity
I let myself be invaded by spite and resentment
We refuse to be judged by someone who has suffered less than us
Here, we regret paradise or anticipate another one, do nothing and go nowhere
The absence of necessity in being born leaves you with a foolish grin
He who hates himself cannot be humble
What is the use of being prized in a world inhabited by madmen, a world mired in mania and stupidity
Salvation in the absolute, ruin in the immediate
From the abyss of birth we emerge to our universal chagrin
Which contradictory impulses should you yield to
What are you waiting for in order to give up
One's work is finished when one can no longer improve it, even though one knows it to be inadequate and incomplete (p. 50)
Whether one succeeds or not comes down to the same thing
Everything is without basis, without substance
Fear is the antidote to boredom, the remedy must be stronger than the disease
Being is suspect, what is to be said of life, which is its deviation and stigma
It is only through religion that one can grasp the many versions of spiritual collapse
The prospect of an imminent end plunges you into a frenzy of activity
Nothing is tragic, everything is unreal
A mind without time to age, to know the long serene disgust of detachment
Everything turns on pain, for we only remember what hurts
Euphorically I suppress word after word from my mind
And it is in those moments that I feel that earth is falling through space and I am falling with it at a nauseating speed
The more you live, the less useful it seems to have lived
Forehead pressed against the pane, staring into the dark
A monster, however horrible, secretly attracts us, pursues us, haunts us
Over the centuries, man has slaved to believe, passing from dogma to dogma, illusion to illusion
I know peace only when my ambitions sleep, once they waken, anxiety repossesses me
The renunciation of the fruit of actions
Every living thing is a source of noise
Your mission is to suffer for all those who suffer without knowing it
Not yet to have digested the affront of being born
Not one moment when I am not external to the universe
Enslavers of the mind
We should repeat to ourselves, every day: I am one of the billions dragging himself across the surface of the earth, one, and no more. This banality justifies any conclusion, any action. (p. 118)
Men will be the victims of death as long as women give birth
Only one thing matters, learning to be the loser
When I rage against this age, I can calm myself merely by thinking about what will happen
The appetite for destruction is so deeply anchored within us that no one manages to extirpate it
With such carrion on my hands, how will I combat the capitulation of my organs
There are all those things which are the lot of the bad monk, indolence, gluttony, the thirst for desolation, greed and aversion for the world, vacillation between tragedy and the equivocal, hope of an inner collapse
It is a great force, to be able to live, without any ambition what ever
It is when I felt nothing when I was closest to the truth
Frequently I lack the strength to confront the daily torture of time
The vile and persuasive refrain: "What do you expect of this world?"
On every occasion we must remember that we are here to make each other wretched
What every man has realised that his birth was a defeat
Original panic in the face of the void
Life seems good only to the madman
Suddenly I am cold, I have no motives left, yet I am living
I cringe before the inventory of my defects
Beware of those who turn their backs on love, ambition, society, for they will take revenge for having renounced
When modes of expression are worn out, art tends toward non-sense, toward a private and incommunicable universe
The troubles of the organs are raised to song
The mind is the great profiteer of the body's defeats, it grows rich at the expense of the flesh it pillages, exulting in its victim's miseries
Man is a wound with nine openings
You can torment yourself at will, unearthing perplexities, obtaining them by every possible means
The cancer of the word
Victims of cynicism, unable to attach themselves to any vice
Every thought should recall the ruin of a smile
I prowl around the depths, draw off certain delirium, like a swindler of the abyss
Nothing slakes my thirst for doubts
I have utterly destroyed within myself, the pride of being human
Death poses a problem which replaces all others
Leading to its conclusion, a depressing train of thought, that affects our entrails, a disaster of the flesh
Embrace the existential delirium, fed on words
A false image of life
In prayer and vomit, in rage or prostration
The genesis of the negative, the venture into the void
Erect, I make a resolution, prone, I revoke it
We no longer dread tomorrow once we learn to take nothingness into our arms
Inside you lies a future alien
My vacillating instincts, my corroded beliefs, my obsessions and aridity
To propagate disequilibrium, to aggravate mental disturbance
The only corpse from which we can gain some advantage is the one preparing itself within us
The moment an animal breaks down it begins resembling a man
You smother me to waken within me the character
Why abandon the game when there remain so many of us to disappoint
Nature has created individuals only to relieve suffering, to help it spread and scatter at their expense
We cannot avoid the defects of men without fleeing from civilisation
The hope is to contradict the future
No action, no success, life is an insect's occupation
The tenacity I have deployed to combat the magic of suicide would have easily sufficed to attain salvation
At the apogee of our disgusts, a rat seems to have crept into our brain to dream there
Your secret coincides with the suffering you crave
I hear the groans of matter, the calls of the inanimate echo through space
It will come the day when man will have understood the accidental nature of his advent and the gratuitousness of his ordeals
Life will be reduced to its just proportions, to a hypothesis of labour
Let one crucial experience intervene in our life and each in his own way makes his entrance into the irrational, wallowing in failure
I have flung myself into confusion until it became my form of piety
Everything persecutes our ideas, beginning with our brain
Toward the antipodes of ecstasy, where the void begins
The prospect of death flings into trances
We raise ourselves towards a purer form of apathy
The flesh spreads further, like a gangrene upon the surface of the globe
To procreate is to love the scourge, to seek to maintain and augment it
You do not yield to the common delirium
At once agent and destroyer of being
Orgy or asceticism
Greedy for obliteration, we enthusiastically accepted our nothingess
The need to wallow in one's insignificance
The greatest of spectacles, uttering dreadful groans in the depths of the abyss
Eager to fail
The flesh is perishable to the point of madness, it is the seat of disease, of incurable nothingness
To vanquish attachments one must contemplate the ultimate nudity of a human being, one's eyes must pierce his entrails and wallow in his secretions, an imminent corpse
Happiness is identified with the destruction of our bonds
The solitary gladly sacrifices decades of his life for a foreboding, for a flash of the absolute
To execrate this world, to discard an illusion, to reduce it to a transitory assemblage of unreal elements
Incapacity for illusion
A mob engulfed in the false, committed to an inferior truth, confusing substance with semblance
Teaching the flesh the art of dissociating desire and act
Suppressing desire is accompanied by a sense of power
A state joyous and without joy
In a fit of unendurable lucidity
It is suffering that gives weight to our thoughts
My negations are made flesh
The organic repugnance, the craving to exist
This is the site of all our attachments
To think is to compete with delirium
Every day we should honour a creature or an object by renouncing them
A reinforcement of our allegiance to the world
The void affords us the certitude of our unreality, the abyss without vertigo
To perceive the throbbing illusion to which human beings are in thrall
To be swollen, stuffed, obese with fear
Nature loathes originality, it rejects and execrates man
You have betrayed your true vocation, which was that of a tyrant or a hermit
To fall asleep with the clear view of one of our defects
Everything blurs and fades in human beings
Fertile for awakening but fatal to action
I question the sincerity of your conversion
To cease to prefer, to occupy yourself with absence, with the obligation to be nothing ever again
He will find himself naked facing the abyss that follows upon all dogmas
The whole world is in flames, the whole world is wrapped in clouds of smoke, the whole world is devoured by fire, the whole world trembles
The contagion of the end
Rotten at the roots
Corroded by wisdom
You wear a mask of objectivity but you strive to suffer, greedy for the worst
Seditious creatures whose every smile is subversive
Our contortions, visible or secret, we communicate to the earth, that vomited us forth
At the climax of our void
The disgust inspired by the sight of any book
Boredom is a higher state
Illusion begets and sustains the world
To stay on this side of madness, you must frequent those more demented than yourself
The thirst for non-becoming
One does not repair chaos
I identify myself with neither my words nor my actions
One disgust, then another, to the point of losing the use of speech and even of the mind
We try to forget the body but it does not forget us
In an orgy of renunciation
I have contributed to the debilitation of the man I was
A hierarchy of unrealities
Man is an abyss
Equilibrium, in all its forms, smothers the mind
Impregnated by that nothing, to the point of intoxication
An emissary of the void, the great tempter, whose schemes are feared
Bliss without substance, without any support in any world at all
Everything seems erased and suspended and I feel nothing
This world was not created according to the will of life
We glimpse the abyss from which sounds the call to prayer
All is transitory
With the appetite for provocation
Even our beliefs are controlled by mechanisms
The nullity of our initiatives
Freely enslaved
To dilute a thought that constitutes its charm
Abandon all comprehension
It is the primordial disorder, our original disease, the ills that overwhelm us
The invariability of human nature
A broken, static vision of the world
Sighs heralding the return of the void
I try to cast the world behind me
All nature is but a concentrated suffering
Unwilling to resign yourself to the future that is forming before your eyes
Your will to illusion knows no boundaries
Abdicate your anxieties in order to bask in the bliss of stagnation
We pray with fervour only in sects, among persecuted minorities, in darkness and in fear
The rage of submission
Men can never be united without a law that deprives them of their will
One must be a priest or a soldier
This vice of our nature delights him
He praises all symbols of authority
A belief's degree of inhumanity guarantees its strength and its duration
A fanatic by his discipline and by method
Lucidity is a permanent martyrdom, an unimaginable tour de force
Music is of an essence superior to life, even superior to death
I live on the inexplicable, gorged with it
The sequence of your failures is so remarkable that it seems to reveal a providential plan
Lucidity is the capacity for non-consent
He shows no haste, follows no direction, generates enthusiasm for no subject
Happiness or misery consists of a certain arrangement of organs
Everything vanishes, silence invades
True health is never felt
The longed for dissolving agent
He is happy only when he contemplates renunciation and prepares himself for it
A destroyer who adds to existence, who enriches by undermining it
The hygienic function of malevolence
Lured by the desert
The more one has suffered, the less one demands
To conceive the act of thought as a poison bath
It is in order to torment ourselves that we are here, and no other reason
To be convinced of the futility of any action, of any destiny
To think neither of the past nor of the future
Life flourishes only through repetition
To recognise in oneself all the vile instincts of man
Every life is the story of a gradual collapse
To sidestep reality (?), to participate in the blissful sleep of objects
Corrupted by knowledge, we have become agents of dissolution
We have been promoted to the rank of incurables, we are matter in pain, shrieking flesh, bones pierced by cries, encroaching upon reality, in strangled lamentation
One articulate prayer, repeated inwardly to the point of hebetude or orgasm, carries more weight than all ideas
To pray unceasingly until your very voice and reason are extinguished
Woe unto those who know they are breathing
I am no-one, I have conquered my name
To hide your life
To join being on the borders of the void
An appetite for unreality
The defeat of the organs
Agent of non-being
Raise yourself to the vision of your (own) nullity
Having nothing to transmit, neutral to the point of abduction, he collapses into well-being, an insignificant state of perfection, of impermeability to death of inattention to oneself and to the world
Suffering without pain
You dream of a series of determined ordeals to free you from intolerable vagueness
Advancing to no purpose
The true weight of beings
Real life begins and ends with agony
Revelation of meaninglessness
A massive deception that hides both life and death
The world chooses to pull down all that is raised up
Restraint, abstention, withdrawal not only from this but from all worlds, a mineral serenity, a preference for petrification, out of fear of both pleasure and pain
After millions of years the great mammals weary of wandering the globe, reaching that degree of explosive lassitude where consciousness sides against itself
A succession without content, time has lost its substance, it is an abstract version of our personal void
Man creates out of freedom a prison
Of cadaverous dignity
With dried-up hands solicited by sanctity and crime, collapsing into prayer and panic
Raised in my eyes to the rank of a revelation
Freed of names and faces
Plunged into subversive prayers, victim of a panic that emanates not from a vision of the world but from the spasms of the flesh
No more ambition, no means of being someone or something, the void incarnate, glands and viscera clairvoyant, bones disabused, a body invaded by lucidity, pure of itself, outside of time
I trace my way back toward sin
It is evidence of my decline
Amorphous masses with neither idols nor ideals, lacking in fanaticism, helpless amid their convulsions
We shall all figure as carrion
Rend my entrails, wring my veins, trample each organ to dust
Knowledge subverts love
You come to loathe your kind precisely because they resemble you
You flourish more in affliction than in joy
Every conviction consists chiefly of hate
An instrument of the flayed man
I renounce my rank in the hierarchy of beings
A throng of hysterics who want another world, here and now
Bloodless, perfect and nil, stripped of sins and vices, with neither depth nor contour, uninitiated into existence
To conceive the void we must execrate all becoming, having endured its weight, its calamity
That murderous curiosity which keeps us from marrying the world
\fi

\end{document}
